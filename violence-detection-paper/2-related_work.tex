\section{Related work}
    \label{sec:relatedwork}
    \subsection{Traditional Methods}
    
    Violence can be detected by traditional methods using hand-crafted feature extraction algorithms and using classical machine learning algorithms such as KNN, AVM, and Adaboost as a classifier. 
    As done by, Harris corner detector~\cite{chen2008recognition}, improved dense trajectory (iDT)~\cite{wang2013action_9}, motion scale-invariant feature transform (MoSIFT)~\cite{xu2014violent_22}, space-time interest points (STIP)~\cite{de2010violence_7}.
    
    Hassner et al.~\cite{hassner2012violent_6} used optical flow magnitude series to detect violence in videos. The features are called violent flow (ViF) descriptors.
    Later, their method was improved by introducing the orientation in the violent flow (ViF) descriptors~\cite{gao2016violence_23}.
    Improved dense trajectory features iDT~\cite{wang2013action_9} remarkably improved the accuracy of human action recognition. 
    This method along with the fisher encoding method extracts significant spatio-temporal features from violent videos~\cite{bilinski2016human_24}.
    
    \subsection{Deep Learning Based Methods}
    Recently, deep learning-based methods gained much interest due to their improved accuracy and better performance than traditional methods. 
    Convolutional neural network (CNN)~\cite{lecun1995convolutional}, 3D convolutional network~\cite{ji_3dconv},  and long short term memory (LSTM) network~\cite{10.1162/neco.1997.9.8.1735} are widely used architecture for the purpose of video understanding~\cite{11convlstm_donahue2015long, c3dtran2015learning, i3dcarreira2017quo}.
    Different deep learning-based approaches have also been used in violence detection~\cite{26_zhou2017violent, 29_sudhakaran2017learning}. 
    	
	Three input streams, RGB frames, optical flow, and accelerated flow maps, were used as input for violence detection~\cite{28_dong2016multi}.
	They used LSTM network to filter out features from the input streams. 
	Using their individual benefits, CNN and LSTM are used together in violent detection~\cite{30_shi2015convolutional,29_sudhakaran2017learning}. 
	The spatial features are extracted by the convolution layers and the mapping of temporal frames was done by LSTM layers.
	FightNet~\cite{26_zhou2017violent} used features from multiple streams, such as RGB frames, optical flow, and acceleration images, and fused them for violence detection. In~\cite{cheng2019rwf}, Flow-Gate network was proposed which used fusion of RGB frames and optical flow.